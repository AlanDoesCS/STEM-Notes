% ----- fancy headers -----
\usepackage{fancyhdr}
\pagestyle{fancy}
\fancyhf{} % clear all

% Page numbers at outer header; titles in inner header
\fancyhead[LE,RO]{\thepage}
\fancyhead[LO]{\nouppercase{\rightmark}} % odd: section/subsection
\fancyhead[RE]{\nouppercase{\leftmark}}  % even: chapter/section
\setlength{\headheight}{14pt}

% Populate header marks nicely (book class)
\renewcommand{\chaptermark}[1]{\markboth{#1}{}}
\renewcommand{\sectionmark}[1]{\markright{\thesection\ #1}}

% --- Make 'plain' pages truly plain (no header/footer) ---
% Used by title page, chapter openings, etc.
\fancypagestyle{plain}{%
	\fancyhf{}%
	% If you want a page number on plain pages, uncomment:
	% \fancyfoot[C]{\thepage}
}

% Optional: one-off style that keeps only a centered footer page number
\fancypagestyle{noheader}{%
	\fancyhf{}%
	\fancyfoot[C]{\thepage}%
}

% Less picky float placement
\makeatletter
\def\fps@figure{!htbp}
\def\fps@table{!htbp}
\makeatother

% Keep floats inside sections (or use [chapter] if you prefer)
\usepackage[section]{placeins}

% Start in front matter (roman numerals)
\AtBeginDocument{\frontmatter}
